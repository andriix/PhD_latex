\documentclass[]{article}

%opening
\title{}
\author{}
\usepackage{neuralnetwork}
\usepackage{amsmath}

\begin{document}

\maketitle

\begin{abstract}
Text of this abstract
\end{abstract}
Main part of the article. An example of an equation:
\[E=mc^2\]
Here, an example of NN can be presented as:

\begin{neuralnetwork}[height=4]
	\newcommand{\x}[2]{$x_#2$}
	\newcommand{\y}[2]{$\hat{y}_#2$}
	\newcommand{\hfirst}[2]{\small $h^{(1)}_#2$}
	\newcommand{\hsecond}[2]{\small $h^{(2)}_#2$}
	\inputlayer[count=3, bias=true, title=Input\\layer, text=\x]
	\hiddenlayer[count=4, bias=false, title=Hidden\\layer 1, text=\hfirst] \linklayers
	\hiddenlayer[count=3, bias=false, title=Hidden\\layer 2, text=\hsecond] \linklayers
	\outputlayer[count=2, title=Output\\layer, text=\y] \linklayers
\end{neuralnetwork}

Also, an example of some system of equation is shown:

\begin{equation*}
	\left\{
	\begin{alignedat}{3}
		% R & L   &  R & L   &  R & L 
		2x & +{} &  y & +{} & 3z & = 10 \\
		x & +{} &  y & +{} &  z & = 6 \\
		x & +{} & 3y & +{} & 2z & = 13
	\end{alignedat}
	\right.
\end{equation*}

The article is finished.
\end{document}
