\documentclass[16pt]{article}
%Packages
\usepackage[T1,T2A]{fontenc}
\usepackage[utf8]{inputenc}
\usepackage[english,ukrainian]{babel}
\usepackage{hyperref}
%Header information
\title{ Dynamical systems, $\epsilon$-machines and complexity }
\author{ Andrii Khrinenko }
\date{}
%Main part
\begin{document}
	\maketitle	

\section{Вступ}

Показник складності Колмогорова є малопридатним для реалізації і практичного застосування. Для оцінки складності можна використовувати епсілон-машини.

* Цікава дисертація про епсілон-машини, використання логістичного тенту та практичне застосування обчислювальної механіки: \href{https://amslaurea.unibo.it/15649/1/mattia_barbaresi_tesi.pdf}{COMPUTATIONAL MECHANICS: from theory to practice} 
	
	
\end{document}